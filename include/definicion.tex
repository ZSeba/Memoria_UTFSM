\section{Definici\'on del Problema}


Ya que ahora, en gran volumen, los usuarios cumplen roles de consumidores y creadores, se crea un ambiente de exceso de información. Un usuario que desea, por ejemplo, ver una película se encuentra con mas de un millón de opciones posibles. Por lo tanto se hace necesaria una forma de reducir las opciones posibles a aquellas que sean más relevantes para el usuario.
Con esta motivación, nace el área de sistemas de recomendación basados en confianza, sustentados en áreas más antiguas de la computación, como los sistemas de recomendación y reputación. Estos sistemas pueden acotar el número de opciones para un usuario y entregarlas en orden de preferencia estimada del usuario.
Ya que cualquier persona puede decir cualquier cosa sobre cualquier cosa, uno de los problemas que se presentan, tanto para un usuario común, como para los sitios web, es que dado el alto volumen de información generada, es difícil determinar su nivel de veracidad (ni las intenciones del usuario que las genera).

\section{Objetivos}

Considerando la problemática recién planteada, se proponen los siguientes objetivos:

\begin{enumerate}

\item [\textbf{Objetivo General}] Utilizar técnicas de computación basada en trust para enriquecer un sistema de recomendación de lugares culturales de la Región Metropolitana
\item [\textbf{Objetivos Específicos}]
	\begin {itemize}
		\item Desarrollar una arquitectura de recomendación, algoritmo y base de datos, que permita enriquecimiento.
		\item Desarrollar una aplicación web que permita aprovechar la arquitectura creada, para entregar lugares culturales de interés al usuario objetivo.
		\item Determinar la relevancia real que tienen los elementos recomendados para el usuario.
	\end{itemize}

\end{enumerate}

\section{Metodología de Trabajo}

Para resolver el problema propuesto se dividirá el proceso en cuatro partes, Investigación, Diseño, Implementación y Validación.

\begin{itemize}
	\item[\textbf{Invesitagión}] Durante esta parte del trabajo, se investigará en detalle los métodos que existen actualmente para resolver problemas similares, principalmente centrándose en sistemas de reputación, recomendación, crowdsourcing y trust. Las fuentes utilizadas serán adjuntas en la sección de bibliografía.
	\item[\textbf{Diseño}] El diseño de la solución se lleva a cabo utilizando lo aprendido durante la investigación y determinando la técnica a utlizar para la resolución del problema. 
	\item[\textbf{Implementación}] Consiste en el desarrollo de la aplicación web que incluya la técnica a utilizar, determinada en la fase de diseño, su fncionamiento se explicará en la sección de Implementación
	\item[\textbf{Validación}] Cuando el algoritmo a utilizar ya está implementado en una aplicación web, se procederá a hacer una prueba con usuarios preliminares, para así evaluar la veracidad de los resultados obtenidos con el algoritmo desarrollado.

\end{itemize}