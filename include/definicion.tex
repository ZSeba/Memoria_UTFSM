\chapter{Definici\'on del Problema}

Actualmente, la colaboración, interacción e intercambio de información son la fuerza que mueve las aplicaciones web de hoy en día (A esto se le conoce como Web 2.0).  Ejemplos de esto son los blogs, redes sociales, wikis y sitios específicamente desarrollados para compartir contenido creado por usuarios, tales como Youtube, Deviantart ,  Twitch o Newgrounds. 
Con el auge de los dispositivos móviles se ha mantenido esta tendencia a
Ya que ahora, en gran volumen, los usuarios cumplen roles de consumidores y creadores, se hace necesario un mecanismo que permita diferenciar a los usuarios y el contenido que producen, para facilitar el enlace entre un usuario y el contenido que éste desea.
Con esta motivación, nace el área de sistemas de recomendación basados en confianza, sustentados en áreas más antiguas, como los sistemas de recomendación y reputación.
Ya que cualquier persona puede decir cualquier cosa sobre cualquier cosa, uno de los problemas que se presentan, tanto para un usuario común, como para los sitios web, es que dado el alto volumen de información generada, es difícil determinar su nivel de veracidad (ni las intenciones del usuario que las genera).
