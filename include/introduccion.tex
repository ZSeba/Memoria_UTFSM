\addcontentsline{toc}{chapter}{Introducci\'on}
\chapter*{Introducci\'on}


\section{Contexto}

Desde su creación en el año 1989, la World Wide Web (www, o W3) ha cambiado las telecomunicaciones permanentemente y ha permitido el intercambio de información entre personas de todo el mundo. Actualmente, la colaboración, interacción e intercambio de información son la fuerza que mueve las aplicaciones web de hoy en día (A esto se le conoce como Web 2.0).  Ejemplos de esto son los blogs, redes sociales, wikis y sitios específicamente desarrollados para compartir contenido creado por usuarios, tales como YouTube\footnote{http://www.youtube.com/} Wikipedia\footnote{http://es.wikipedia.org/} ,  Yelp \footnote{http://www.yelp.com/} o  TripAdvisor\footnote{http://www.tripadvisor.com/}. 
Durante la segunda mitad de la década de los años 2000 hasta ahora se ha mantenido esta tendencia hacia el usuario como la principal fuente de generación de contenido, lo cual se conoce como \" Crowdsourcing \". En el caso de TripAdvisor, el crowsourcing ha significado una plataforma con una cantidad enorme de reseñas y opiniones de usuarios en todo el mundo. Wikipedia, por su parte, ha utilizado el crowsourcing para crear la enciclopedia más grande y completa de todos los tiempos, con más de 5 millones de artículos (CITA) , en su versión en inglés.  El mayor valor de este paradigma ha dado acceso a información que habría sido imposible de generar con el modelo Web 1.0 en el que creador y consumidor eran personas distintas. 